\usepackage{xparse}
\newcommand{\bnm}{\begin{newmath}}
\newcommand{\enm}{\end{newmath}}

\newcommand{\bea}{\begin{eqnarray*}}%
\newcommand{\eea}{\end{eqnarray*}}%

\newcommand{\bne}{\begin{newequation}}
\newcommand{\ene}{\end{newequation}}

\newcommand{\bal}{\begin{newalign}}
\newcommand{\eal}{\end{newalign}}

\newenvironment{newalign}{\begin{align}%
\setlength{\abovedisplayskip}{4pt}%
\setlength{\belowdisplayskip}{4pt}%
\setlength{\abovedisplayshortskip}{6pt}%
\setlength{\belowdisplayshortskip}{6pt} }{\end{align}}

\newenvironment{newmath}{\begin{displaymath}%
\setlength{\abovedisplayskip}{4pt}%
\setlength{\belowdisplayskip}{4pt}%
\setlength{\abovedisplayshortskip}{6pt}%
\setlength{\belowdisplayshortskip}{6pt} }{\end{displaymath}}

\newenvironment{neweqnarrays}{\begin{eqnarray*}%
\setlength{\abovedisplayskip}{-4pt}%
\setlength{\belowdisplayskip}{-4pt}%
\setlength{\abovedisplayshortskip}{-4pt}%
\setlength{\belowdisplayshortskip}{-4pt}%
\setlength{\jot}{-0.4in} }{\end{eqnarray*}}

\newenvironment{newequation}{\begin{equation}%
\setlength{\abovedisplayskip}{4pt}%
\setlength{\belowdisplayskip}{4pt}%
\setlength{\abovedisplayshortskip}{6pt}%
\setlength{\belowdisplayshortskip}{6pt} }{\end{equation}}


\newcounter{ctr}
\newcounter{savectr}
\newcounter{ectr}

\newenvironment{newitemize}{%
\begin{list}{\mbox{}\hspace{5pt}$\bullet$\hfill}{\labelwidth=15pt%
\labelsep=4pt \leftmargin=12pt \topsep=3pt%
\setlength{\listparindent}{\saveparindent}%
\setlength{\parsep}{\saveparskip}%
\setlength{\itemsep}{3pt} }}{\end{list}}


\newenvironment{newenum}{%
\begin{list}{{\rm (\arabic{ctr})}\hfill}{\usecounter{ctr} \labelwidth=17pt%
\labelsep=5pt \leftmargin=22pt \topsep=3pt%
\setlength{\listparindent}{\saveparindent}%
\setlength{\parsep}{\saveparskip}%
\setlength{\itemsep}{2pt} }}{\end{list}}

%%%%%%%%%%%%%%%%%%%%%%%%%%%%%%%%%%%%%%%%%%%%%%%%%%%%%%%%%%%%%%%%%%%%%%%%%%%%%%
%
% Figure and table macros
%

\newcounter{mytable}
\def\mytable{\begin{centering}\refstepcounter{mytable}}
\def\endmytable{\end{centering}}

\def\mytablecaption#1{\vspace{2mm}
  \centerline{Table \arabic{mytable}.~{#1}}
  \vspace{6mm}
  \addcontentsline{lot}{table}{\protect\numberline{\arabic{mytable}}~{#1}}
}


\newcounter{myfig}
\def\myfig{\begin{centering}\refstepcounter{myfig}}
\def\endmyfig{\end{centering}}

\def\myfigcaption#1{
             \vspace{2mm}
             \centerline{\textsf{Figure \arabic{myfig}.~{#1}}}
             \vspace{6mm}
             \addcontentsline{lof}{figure}{\protect\numberline{\arabic{myfig}}~{#1}}}


\newlength{\saveparindent}
\setlength{\saveparindent}{\parindent}
\newlength{\saveparskip}
\setlength{\saveparskip}{\parskip}

\newcommand{\decOracle}{\textbf{Dec}}

\newcommand{\negsmidge}{{\hspace{-0.1ex}}}
\newcommand{\cdotsm}{\negsmidge\negsmidge\negsmidge\cdot\negsmidge\negsmidge\negsmidge}

\def\suchthatt{\: :\:}
\newcommand{\E}{{\rm I\kern-.3em E}}
\newcommand{\Prob}[1]{{\Pr\left[\,{#1}\,\right]}}
\newcommand{\Probb}[2]{{\Pr}_{#1}\left[\,{#2}\,\right]}
\newcommand{\CondProb}[2]{{\Pr}\left[\: #1\:\left|\right.\:#2\:\right]}
\newcommand{\CondProbb}[2]{\Pr[#1|#2]}
\newcommand{\ProbExp}[2]{{\Pr}\left[\: #1\:\suchthatt\:#2\:\right]}
\newcommand{\Ex}[1]{{\textnormal{E}\left[\,{#1}\,\right]}}
\newcommand{\Exx}{{\textnormal{E}}}
\newcommand{\given}{\ensuremath{\,\big|\,}}


\newcommand{\true}{\mathsf{true}}
\newcommand{\false}{\mathsf{false}}
\def\negl{\mathsf{negl}}


\newcommand{\secref}[1]{\mbox{Section~\ref{#1}}}
\newcommand{\appref}[1]{\mbox{Appendix~\ref{#1}}}
\newcommand{\thref}[1]{\mbox{Theorem~\ref{#1}}}
\newcommand{\defref}[1]{\mbox{Definition~\ref{#1}}}
\newcommand{\corref}[1]{\mbox{Corollary~\ref{#1}}}
\newcommand{\lemref}[1]{\mbox{Lemma~\ref{#1}}}
\newcommand{\clref}[1]{\mbox{Claim~\ref{#1}}}
\newcommand{\propref}[1]{\mbox{Proposition~\ref{#1}}}
\newcommand{\factref}[1]{\mbox{Fact~\ref{#1}}}
\newcommand{\remref}[1]{\mbox{Remark~\ref{#1}}}
\newcommand{\figref}[1]{\mbox{Figure~\ref{#1}}}
\renewcommand{\algref}[1]{\mbox{Algorithm~\ref{#1}}}
% \newcommand{\eqref}[1]{\mbox{Equation~(\ref{#1})}}
% Have to use \renewcommand because exists already in amsmath
\renewcommand{\eqref}[1]{\mbox{Equation~(\ref{#1})}}
\newcommand{\consref}[1]{\mbox{Construction~\ref{#1}}}
\newcommand{\tabref}[1]{\mbox{Table~\ref{#1}}}

\newcommand{\get}{{\:{\leftarrow}\:}}
\newcommand{\gett}[1]{\:{\leftarrow}_{#1}\:}
\newcommand{\getsr}{{\:{\leftarrow{\hspace*{-3pt}\raisebox{.75pt}{$\scriptscriptstyle\$$}}}\:}}
\newcommand{\getm}{{\:\leftarrow_{\mdist}\:}}
\newcommand{\getd}{{\:\leftarrow_{\ddist}\:}}
%\newcommand{\getm}{{\:{\leftarrow{\hspace*{-3pt}\raisebox{.75pt}{$\scriptscriptstyle \mdist$}}}\:}}
\newcommand{\getk}{{\:\leftarrow_{\kdist}\:}}
%\newcommand{\getk}{{\:{\leftarrow{\hspace*{-3pt}\raisebox{.75pt}{$\scriptscriptstyle \kdist$}}}\:}}
\newcommand{\getp}{{\:\leftarrow_{p}\:}}



\newcommand{\gamesfontsize}{\small}
\newcommand{\fpage}[2]{\framebox{\begin{minipage}[t]{#1\textwidth}\setstretch{1.1}\gamesfontsize  #2 \end{minipage}}}
\newcommand{\mpage}[2]{\begin{minipage}[t]{#1\textwidth}\setstretch{1.1}\gamesfontsize  #2 \end{minipage}}

\newcommand{\hpages}[3]{\begin{tabular}{cc}\begin{minipage}[t]{#1\textwidth} #2 \end{minipage} & \begin{minipage}[t]{#1\textwidth} #3 \end{minipage}\end{tabular}}

\newcommand{\hpagess}[4]{
        \begin{tabular}[t]{c@{\hspace*{.5em}}c}
        \begin{minipage}[t]{#1\textwidth}\gamesfontsize #3 \end{minipage}
        &
        \begin{minipage}[t]{#2\textwidth}\gamesfontsize #4 \end{minipage}
        \end{tabular}
    }

\newcommand{\hpagesss}[6]{
        \begin{tabular}[t]{c@{\hspace*{.5em}}c@{\hspace*{.5em}}c@{\hspace*{.5em}}c}
        \begin{minipage}[t]{#1\textwidth}\gamesfontsize #4 \end{minipage}
        &
        \begin{minipage}[t]{#2\textwidth}\gamesfontsize #5 \end{minipage}
        &
        \begin{minipage}[t]{#3\textwidth}\gamesfontsize #6 \end{minipage}
        \end{tabular}
    }

\newcommand{\hpagessss}[8]{
        \begin{tabular}{c@{\hspace*{.5em}}c@{\hspace*{.5em}}c@{\hspace*{.5em}}c}
        \begin{minipage}[t]{#1\textwidth}\gamesfontsize #5 \end{minipage}
        &
        \begin{minipage}[t]{#2\textwidth}\gamesfontsize #6 \end{minipage}
        &
        \begin{minipage}[t]{#3\textwidth}\gamesfontsize #7 \end{minipage}
        &
        \begin{minipage}[t]{#4\textwidth}\gamesfontsize #8 \end{minipage}
        \end{tabular}
    }


\newcommand{\hfpages}[3]{\hfpagess{#1}{#1}{#2}{#3}}
\newcommand{\hfpagess}[4]{
        \begin{tabular}[t]{c@{\hspace*{.5em}}c}
        \framebox{\begin{minipage}[t]{#1\textwidth}\setstretch{1.1}\gamesfontsize #3 \end{minipage}}
        &
        \framebox{\begin{minipage}[t]{#2\textwidth}\setstretch{1.1}\gamesfontsize #4 \end{minipage}}
        \end{tabular}
    }
\newcommand{\hfpagesss}[6]{
        \begin{tabular}[t]{c@{\hspace*{.5em}}c@{\hspace*{.5em}}c}
        \framebox{\begin{minipage}[t]{#1\textwidth}\setstretch{1.1}\gamesfontsize #4 \end{minipage}}
        &
        \framebox{\begin{minipage}[t]{#2\textwidth}\setstretch{1.1}\gamesfontsize #5 \end{minipage}}
        &
        \framebox{\begin{minipage}[t]{#3\textwidth}\setstretch{1.1}\gamesfontsize #6 \end{minipage}}
        \end{tabular}
    }
\newcommand{\hfpagessss}[8]{
        \begin{tabular}[t]{c@{\hspace*{.5em}}c@{\hspace*{.5em}}c@{\hspace*{.5em}}c}
        \framebox{\begin{minipage}[t]{#1\textwidth}\setstretch{1.1}\gamesfontsize #5 \end{minipage}}
        &
        \framebox{\begin{minipage}[t]{#2\textwidth}\setstretch{1.1}\gamesfontsize #6 \end{minipage}}
        &
        \framebox{\begin{minipage}[t]{#3\textwidth}\setstretch{1.1}\gamesfontsize #7 \end{minipage}}
        &
        \framebox{\begin{minipage}[t]{#4\textwidth}\setstretch{1.1}\gamesfontsize #8 \end{minipage}}
        \end{tabular}
    }

\newcommand{\vecw}{\mathbf{w}}
\newcommand{\R}{\mathbb{R}}
\newcommand{\N}{\mathbb{N}}
\newcommand{\Z}{\mathbb{Z}}
\newcommand{\load}{L}
\newcommand{\coll}{\mathsf{Coll}}
\newcommand{\nocoll}{\overline{\mathsf{Coll}}}


\newcommand{\Img}{\textsf{Img}}

%%%%%%%%%%%%%%%%%%%%%%%%%%%%%%%%%%%%%%%%%%%%%%%%%%%%%%%%%%%%%%%%%%%%%%%%%%%%%%%%
%%%% Fonts and symbols
%%%%%%%%%%%%%%%%%%%%%%%%%%%%%%%%%%%%%%%%%%%%%%%%%%%%%%%%%%%%%%%%%%%%%%%%%%%%%%%%
\newcommand\funcfont{\textsf}
\newcommand\variablefont{\texttt}

%%%%%%%%%%%%%%%%%%%%%%%%%%%%%%%%%%%%%%%%%%%%%%%%%%%%%%%%%%%%%%%%%%%%%%%%%%%%%%%%
%%%%%%%%%%%%%%%%%%%%%%%%%%%%%%%% NEW COMMANDS %%%%%%%%%%%%%%%%%%%%%%%%%%%%%%%%%%
%%%%%%%%%%%%%%%%%%%%%%%%%%%%%%%%%%%%%%%%%%%%%%%%%%%%%%%%%%%%%%%%%%%%%%%%%%%%%%%%

\def \Perm {\funcfont{Perm}}
\def \calC {{\mathcal{C}}}
\def \calU {{\mathcal{U}}}
\renewcommand{\u}{\ensuremath{u}}
\newcommand{\unew}{\ensuremath{\tilde{u}}}

\newcommand{\calN}{\mathcal{N}}
\def \sspace {{\mathcal{S}}}
\def \strings {{\mathcal{S}}}
\def \slen {{s}}
\def \kspace {{\mathcal{K}}}
\def \kspacesize {{m}}
\def \mspacesize {{n}}
\def \kdict {D}
\def \dictsize {d}
\newcommand{\kdist}{p_k}
\newcommand{\mdist}{\ensuremath{{W}}}
\newcommand{\alldist}{\rho}
\newcommand{\pwdist}{\transgen}
\newcommand{\ddist}{\rho_{dec}}
\newcommand{\PWset}{{\mathcal{P}}}  % TODO: fix, same as \pwdist
\newcommand{\PWsetvec}{\vec{\mathcal{P}}}
\newcommand{\PWvec}{\vec{P}}
\newcommand{\domvec}{\vec{D}}
\newcommand{\humanornot}{\vec{h}}
\newcommand{\dom}{\textsf{dom}}
%\def \kdist {{\kappa}}
%\def \mdist {{\mu}}
%\def \ddist {{\delta}}
\def \pspace {{\mathcal{P}}}
\def \mpspace {{\mathcal{MP}}}
\def \cspace {{\mathcal{C}}}
\def \key {\kappa}
\def \msg {M}
\def \seed {S}
\def \ctxt {C}
\def \ctxtpart {C_2}
\newcommand{\genprime}{{\textsf{GenPrime}}}
\newcommand{\isprime}{{\textsf{IsPrime}}}
\newcommand{\LeastLesserPrime}{{\textsf{PrevPrime}}}
\newcommand{\pwset}{\mathcal{S}}
\newcommand{\DTE}{{\textsf{DTE}}}
\newcommand{\encode}{{\textsf{encode}}}
\newcommand{\decode}{{\textsf{decode}}}

\newcommand{\DTEsingle}{{\textsf{1PW-DTE}}}
\newcommand{\encodesingle}{{\textsf{1PW-encode}}}
\newcommand{\decodesingle}{{\textsf{1PW-decode}}}

\newcommand{\DTErss}{{\textsf{RSS-DTE}}}
\newcommand{\encoderss}{{\textsf{RSS-encode}}}
\newcommand{\decoderss}{{\textsf{RSS-decode}}}

\newcommand{\DTEindep}{{\textsf{MPW-DTE}}}
\newcommand{\encodeindep}{{\textsf{MPW-encode}}}
\newcommand{\decodeindep}{{\textsf{MPW-decode}}}


\newcommand{\DTEsub}{{\textsf{SG-DTE}}}
\newcommand{\encodesub}{{\textsf{SG-encode}}}
\newcommand{\decodesub}{{\textsf{SG-decode}}}
\newcommand{\decodekamf}{{\textsf{KAMF-decode}}}
\newcommand{\DTEis}{{\textsf{IS-DTE}}}
\newcommand{\encodeis}{{\textsf{is-encode}}}
\newcommand{\decodeis}{{\textsf{is-decode}}}
\newcommand{\DTErej}{{\textsf{REJ-DTE}}}
\newcommand{\encoderej}{{\textsf{rej-encode}}}
\newcommand{\decoderej}{{\textsf{rej-decode}}}
\newcommand{\DTErsarej}{{\textsf{RSA-REJ-DTE}}}
\newcommand{\encodeRSAREJ}{{\textsf{rsa-rej-encode}}}
\newcommand{\decodeRSAREJ}{{\textsf{rsa-rej-decode}}}
\newcommand{\DTErsainc}{{\textsf{RSA-INC-DTE}}}
\newcommand{\encodeRSAINC}{{\textsf{rsa-inc-encode}}}
\newcommand{\decodeRSAINC}{{\textsf{rsa-inc-decode}}}
\newcommand{\DTEunf}{{\textsf{UNF-DTE}}}
\newcommand{\DTEnunf}{{\textsf{NUNF-DTE}}}


%\newcommand{\encodeis}{{\textsf{encode}_{\textrm{is}}}}
%\newcommand{\decodeis}{{\textsf{decode}_{\textrm{is}}}}
\newcommand{\rep}{\textsf{rep}}
\newcommand{\isErr}{\epsilon_{\textnormal{is}}}
\newcommand{\incErr}{\epsilon_{\textnormal{inc}}}
\def \enc {{\textsf{E}}}
\def \dec {{\textsf{D}}}
\def \SEscheme {{\textsf{SE}}}
\def \HEscheme {{\textsf{HE}}}
\def \CTR {{\textsf{CTR}}}
\def \encHE {{\textsf{HEnc}}}
\def \HIDE {{\textsf{HiaL}}}
\def \encHIDE {{\textsf{HEnc}}}
\def \decHIDE {{\textsf{HDec}}}
\def \decHE {{\textsf{HDec}}}
\def \encHEt {{\textsf{HEnc2}}}
\def \decHEt {{\textsf{HDec2}}}

\newcommand{\myind}{\hspace*{1em}}
\newcommand{\thh}{^{\textit{th}}} % th
\newcommand{\concat}{\,\|\,}
\newcommand{\dotdot}{..}
\newcommand{\emptystr}{\varepsilon}

\newcommand{\round}{\textsf{round}}

\newcommand{\alphabar}{\overline{\alpha}}
\newcommand{\numbinsbar}{\overline{b}}
\newcommand{\numballs}{a}
\newcommand{\numbins}{b}

%\def \encHE {{\sf{enc}^{HE}}}
%\def \decHE {{\sf{dec}^{HE}}}
%\def \encHEt {{\sf{enc}^{HE2}}}
%\def \decHEt {{\sf{dec}^{HE2}}}
\def \idealHE {{\mathcal{HE}}}
\def \IEnc {{\mathbf{\rho}}}
\def \IDec {{\mathbf{\rho^{-1}}}}
\def \OEnc {{\mathbf{Enc}}}
\def \ODec {{\mathbf{Dec}}}
\newcommand{\SimuProc}{\mathbf{Sim}}
\newcommand{\ROProc}{\mathbf{RO}}
\newcommand{\PrimProc}{\mathbf{Prim}}
\def \stm {g}
\def \istm {\hat{g}}
\def \kts {{f}}
\def \lex {{\sf lex}}
\def \part {part}
\def \kd {{\sf{kd}}}
\def \msgdist {{d}}
\def \keydist {{r}}
\def \ind {{\sf{index}}}
\def \kprf {z}
\def \adv {{\mathcal A}}
\def \pwds {u}
\newcommand{\mpw}{mpw}
\newcommand{\pw}{w}
\newcommand{\pwvec}{\vec{\pw}}
\newcommand{\vecx}{\vec{x}}
\def \tokens {v}
\def \calP{{\mathcal{P}}}
\def \template{{\mathcal{T}}}
\def \vaultset{{\mathcal{V}}}
\def \ext {{\sf ext}}
\def \offset {\delta}
\def \maxweight {\epsilon}
\def \advo {{\mathcal{A}}^{*}}

\newcommand{\Chall}{\textsf{Ch}}
\newcommand{\Test}{\textnormal{\textsf{Test}}}
\newcommand{\RoR}{\textsf{RoR}}
\newcommand{\MI}{\textnormal{MI}}
\providecommand{\MR}{\textnormal{MR}}
\newcommand{\MRCCA}{\textnormal{MR-CCA}}
\newcommand{\SAMP}{\textnormal{SAMP}}
\newcommand{\DTEgame}{\textnormal{SAMP}}
\newcommand{\KR}{\textnormal{KR}}
\newcommand{\advA}{{\mathcal{A}}}
\newcommand{\advR}{{\mathcal{R}}}
\newcommand{\advB}{{\mathcal{B}}} % 
\newcommand{\advC}{{\mathcal{C}}} % C
\newcommand{\advD}{{\mathcal{D}}} % D
\newcommand{\advE}{{\mathcal{E}}}
\newcommand{\advF}{{\mathcal{F}}}
\newcommand{\advG}{{\mathcal{G}}}
\newcommand{\advI}{{\mathcal{I}}}
\newcommand{\nextval}{\;;\;}
\newcommand{\TabC}{\texttt{C}}
\newcommand{\TabR}{\texttt{R}}
\newcommand{\Hash}{H}
\newcommand{\Cipher}{\pi}
\newcommand{\CipherInv}{\pi^{-1}}
\newcommand{\simu}{{\mathcal S}}
\newcommand{\prim}{P}
\newcommand{\maxguess}{\gamma}


\newcommand{\bigO}{\mathcal{O}}
\newcommand{\calG}{{\mathcal{G}}}

\def\sqed{{\hspace{5pt}\rule[-1pt]{3pt}{9pt}}}
\def\qedsym{\hspace{2pt}\rule[-1pt]{3pt}{9pt}}

\newcommand{\Colon}{{\::\;}}
\newcommand{\good}{\textsf{Good}}

\newcommand\Tvsp{\rule{0pt}{2.6ex}}
\newcommand\Bvsp{\rule[-1.2ex]{0pt}{0pt}}
\newcommand{\TabPad}{\hspace*{5pt}}
\newcommand\TabSep{@{\hspace{5pt}}|@{\hspace{5pt}}}
\newcommand\TabSepLeft{|@{\hspace{5pt}}}
\newcommand\TabSepRight{@{\hspace{5pt}}|}


\DeclareMathOperator*{\argmin}{argmin}
\DeclareMathOperator*{\argmax}{argmax}
\newcommand{\comma}{\textnormal{,}}

\renewcommand{\paragraph}[1]{\vspace*{6pt}\noindent\textbf{#1}\;}

\newcommand{\weirvault}{\textsf{Pastebin}\xspace}
\newcommand{\ndssvault}{\textsf{DBCBW}\xspace}




\newcommand{\reminder}[1]{ [[[ \marginpar{\mbox{$<==$}} #1 ]]] }

%
% New theorem types: (Already in CCS template)
%
\newtheorem{observation}{Observation}
%\newtheorem{definition}{Definition}
\newtheorem{claim}{Claim}
\newtheorem{assumption}{Assumption}
\newtheorem{fact}{Fact}
% \newtheorem{theorem}{Theorem}[section]
% \newtheorem{lemma}{Lemma}[section]
% \newtheorem{corollary}{Corollary}[section]
% \newtheorem{proposition}{Proposition}
% \newtheorem{example}{Example}

%
% Definitions:
%
\def \blackslug{\hbox{\hskip 1pt \vrule width 4pt height 8pt
    depth 1.5pt \hskip 1pt}}
\def \qed{\quad\blackslug\lower 8.5pt\null\par}
% In-line QED, for ending a proof with a $$ formula
% In-line QED, for ending a proof with a $$ formula
\def \inQED{\quad\quad\blackslug}
\def \Qed{\QED}
\def \QUAD{$\Box$}
\def \Proof{\par\noindent{\bf Proof:~}}
\def \proof{\Proof}
\def \poly {\mbox{$\mathsf{poly}$}}
\def \binary {\mbox{$\mathsf{binary}$}}
\def \ones {\mbox{$\mathsf{ones}$}}
\def \rank {\mbox{$\mathsf{rank}$}}
\def \bits {\mbox{$\mathsf{bits}$}}
\def \factorial {\mbox{$\mathsf{factorial}$}}
\def \fr {\mbox{$\mathsf{fr}$}}
\def \pr {\mbox{$\mathsf{pr}$}}
\def \zon {\{0,1\}^n}
\def \zo  {\{0,1\}}
\def \zok {\{0,1\}^k}
\def \mo {s}


\def\utilcnt{\ensuremath{\mu_{\mathrm{cnt}}}}
\def\utiltime{\ensuremath{\mu_{\mathrm{time}}}}
\def\ex{\ensuremath{{\mathrm{ex}}}}
\def\rlx{\ensuremath{{\mathrm{rlx}}}}
\def\tp{\textsf{TP}}
\def\cp{\textnormal{\textsf{CP}}\xspace}
\def\edistcutoff{\edist}
\def\entcutoff{\ensuremath{m}}
\def\relentcutoff{{\sigma}}
\def\mutt{\mu_{\mathrm{tt}}}

\newcommand{\Hdot}{H(\mbox{ } \cdot \mbox{ }  , \mbox{ } \del)}


\newcounter{mynote}[section]
\newcommand{\notecolor}{blue}
\newcommand{\thenote}{\thesection.\arabic{mynote}}
\newcommand{\tnote}[1]{\refstepcounter{mynote}{\bf \textcolor{\notecolor}{$\ll$TomR~\thenote: {\sf #1}$\gg$}}}

\newcommand{\fixme}[1]{{\textcolor{red}{[FIXME: #1]}}}
\newcommand{\todo}[1]{{\textcolor{red}{[TODO: #1]}}}


\newcommand\ignore[1]{}


\newcommand\simplescheme{simple}


\newcommand{\KDF}{\mathsf{KDF}}
\newcommand{\salt}{\mathsf{sa}}
\newcommand{\PRF}{F}
\newcommand{\subgram}{\mathsf{SG}}
\newcommand{\popdomains}{\mathcal{D}}

\newcommand{\retrieve}{\textsf{Sync}}
\newcommand{\update}{\textsf{Insert}}

\newcommand{\dictW}{\textbf{D1}\xspace}
\newcommand{\dictF}{\textbf{D2}\xspace}

\newcommand{\str}{\text{str}}
\newcommand{\calS}{{\mathcal S}}

% \newcommand{\new}[1]{\textcolor{red}{\sf #1}}
\newcommand{\new}[1]{#1}


%% ------------------------- Rahul -----------------------
\newcounter{rcnote}[section]
\newcommand{\rcthenote}{\thesection.\arabic{rcnote}}
\newcommand{\rcnote}[1]{\refstepcounter{rcnote}{\bf \textcolor{magenta}{$\ll$RC~\rcthenote: {\sf #1}$\gg$}}}

\newcounter{mrnote}[section]
\newcommand{\mrthenote}{\thesection.\arabic{mrnote}}
\newcommand{\mrnote}[1]{\refstepcounter{mrnote}{\bf \textcolor{green}{$\ll$MR~\mrthenote: {\sf #1}$\gg$}}}

\newcounter{fknote}[section]
\newcommand{\fkthenote}{\thesection.\arabic{fknote}}
\newcommand{\fknote}[1]{\refstepcounter{fknote}{\bf \textcolor{blue}{$\ll$FK~\fkthenote: {\sf #1}$\gg$}}}

\newcounter{anote}[section]
\newcommand{\ajthenote}{\thesection.\arabic{anote}}
\newcommand{\anote}[1]{\refstepcounter{anote}{\bf \textcolor{cyan}{$\ll$AJ~\ajthenote: {\sf #1}$\gg$}}}



\newcommand{\mytab}{\hspace*{.4cm}}
\def\half{{1\over 2}}
\newcommand{\NT}[1]{\texttt{#1}}
\DeclareMathSymbol{\mlq}{\mathord}{operators}{``}
\DeclareMathSymbol{\mrq}{\mathord}{operators}{`'}
\newcommand{\calO}{{\mathcal O}}
\newcommand{\calA}{{\mathcal A}}
\newcommand{\kamfplus}{Kamouflage\textbf{+}\xspace}
% \newcommand{\genfrom}[1]{\;{\stackrel{\,#1}{\leftarrow}}\;}
\newcommand{\genfrom}[1]{{\:{\leftarrow{\hspace*{-3pt}\raisebox{.75pt}{$\scriptscriptstyle#1$}}}\:}}
%\newcommand{\genfrom}[1]{\;\leftarrow{\tiny \$} #1\;}
\newcommand{\twopartdef}[4]
{
  \left\{
    \begin{array}{ll}
      #1 & \mbox{if } #2 \\[4pt]
      #3 & #4
      \end{array}
      \right.
}
\newcommand{\threepartdef}[6]
{
  \left\{
    \begin{array}{lll}
      #1 & \mbox{if } #2 \\
      #3 & \mbox{if } #4 \\
      #5 & \mbox{if } #6
      \end{array}
      \right.
}

\newcommand{\gt}[1]{\gamma_{#1,\maxdist}}
\newcommand{\gmt}[2]{\gamma_{#1,#2}}
\def\nh{\ensuremath{N}}
\def\ball{\ensuremath{B}}
\def\anh{\ensuremath{\tilde{N}_k}}
% \newcommand{\nh}[2]{{N_{#1}(#2)}}

\newcommand{\ballsizet}[1]{{\beta_{#1,\maxdist}}}
\newcommand{\ballsize}[2]{{\beta_{#1,#2}}}
\newcommand{\rh}[2]{{\bf R}_{#1, #2}}
\newcommand{\rhf}[2]{R_{f, \gamma}}
\newcommand{\realm}{{m}}
% \newcommand{\inputm}{{\tilde{m}}}
\newcommand{\lmid}{\ell_{\realm, m'}}
\newcommand{\cipherlength}{n}
\renewcommand{\SS}{\textsf{SS}}
\newcommand{\Rec}{\textsf{Rec}\xspace}
\newcommand{\rec}{\textsf{rec}\xspace}
\newcommand{\Rep}{\textsf{Rep}\xspace}
\newcommand{\Gen}{\textsf{Gen}}
\newcommand{\dis}{\textsf{dis}}

\def\chk{\textnormal{\textsf{Chk}}\xspace}
\def\reg{\textnormal{\textsf{Reg}}\xspace}
\def\exchk{\textnormal{\textsf{ExChk}}\xspace}
\def\adpchk{\textnormal{\textsf{AdpChk}}\xspace}
\def\RKROR{\textnormal{MKROR}\xspace}
\def\SRKROR{\textnormal{SKROR}\xspace}
\def\ROR{\textnormal{ROR}\xspace}
\def\ROBUST{\textnormal{ROB}\xspace}
\def\OFFDIST{\textnormal{OFFDIST}\xspace}
\def\POFFDIST{\overline{\textnormal{OFFDIST}}\xspace}
\def\OFFGUESS{\textnormal{OFFGUESS}\xspace}
\def\ONGUESS{\overline{\textnormal{ONGUESS}}\xspace}
\def\ONATTACK{\textnormal{ONATTACK}\xspace}
\def\adp{\ensuremath{\Pi\xspace}}
\def\Check{\textnormal{\textsf{Checker}}}
\def\CheckPtxt{\textnormal{\textsf{PChecker}}}
\def\trans{\textnormal{\ensuremath{T}}}
\def\transgen{\textnormal{\ensuremath{\mathcal{T}}}}
\def\pke{\ensuremath{\mathcal{E}}}
\def\pkenc{\ensuremath{\pke^{\mathrm{enc}}}}
\def\pkdec{\ensuremath{\pke^{\mathrm{dec}}}}
\def\sH{\ensuremath\mathrm{H}^s}
\def\H{\ensuremath{\mathsf{H}}\xspace}
\def\T{\ensuremath{\mathsf{T}}\xspace}
\def\W{\ensuremath{\mathsf{W}}\xspace}
\def\F{\ensuremath{\mathsf{F}}\xspace}
\def\ret{\ensuremath{\mathrm{return}}\xspace}
\def\cs{\ensuremath{t}} % Cache Size
\def\waitlen{\ensuremath{\omega}\xspace} %waitlist size
\newcommand{\M}{{\mathcal{M}}}
\newcommand{\ent}{{\bf H}}
\newcommand{\Hfuzzy}{{\mathrm{H}}^{\rm fuzz}}
\newcommand{\Hminfuzzy}{{\mathrm{H}}^{{\rm{fuzz}}}_{t,\infty}}
\newcommand{\Hpfuzzy}{{\mathrm{H}}^{{\rm{pfuzz}}}}
\newcommand{\Hinf}{{\bf H}^{{\rm \min}}_{\infty}}
\newcommand{\Hminpfuzzy}{{\bf H}^{{\rm pfuzz}}_{t,\infty}}


\def\sa{{\sf sa}}
\def\err{{\varepsilon}}%^{(e)}}}
\newcommand{\dist}{\mathcmd{p}}
\newcommand{\distest}{\hat{\mathcmd{p}}}
\newcommand{\distvec}{\mathbf{\dist}}
\newcommand{\distvecest}{\hat{\mathbf{p}}}
\newcommand{\w}{{w}}
\newcommand{\wnew}{\tilde{w}}
\newcommand{\m}{{w}}
\newcommand{\mnew}{{\tilde{m}}}
\newcommand{\mspacevec}{\mathcmd{\mathbf{W}}}
\newcommand{\mvec}{\mathcmd{\mathbf{\m}}}
\newcommand{\mvecss}[1]{\mvec_1\ldots\mvec_{#1}}
\renewcommand{\mspace}{\mathcmd{W}}
\newcommand{\mspacebot}{{\mspace}_\bot}

\newcommand{\similar}{\mathrm{sim}}
\newcommand{\similarvec}{\mathrm{\mathbf{sim}}}

\def \mvecnew{{\tilde{\mvec}}}
\newcommand{\mvecnewss}[1]{\mvecnew_1\ldots\mvecnew_{#1}}

\DeclareDocumentCommand{\edist}{o o}{
  \ensuremath{
    \IfNoValueTF{#1}{{d}}{{\sf d}(#1,#2)}
  }
}

\newcommand{\utilinc}{\mu}
\newcommand{\secloss}{\Delta}
\newcommand{\seclossg}{\Delta^\textnormal{greedy}}
\newcommand{\seclosso}{\Delta}
%\newcommand{\maxlambda}{\lambda^*}
%\newcommand{\maxfuzzlambda}{\tilde{\lambda}^*}
\newcommand{\fuzzlambda}{\lambda^\textnormal{fuzzy}}
\newcommand{\greedylambda}{\lambda^\textnormal{greedy}}
\newcommand{\greedylambdaon}{\tilde{\lambda}^\mathrm{on}}
\newcommand{\greedylambdaoff}{\tilde{\lambda}^\mathrm{off}}
\newcommand{\seclosson}{\secloss^\mathrm{on}}
\newcommand{\seclossoff}{\secloss^\mathrm{off}}
\def \edit {\ensuremath{e}}
\newcommand\error{e}
\newcommand\tab[1][1cm]{\hspace*{#1}}

\newcommand{\mathcmd}[1]{\ensuremath{#1}\xspace} % to use a command both in math mode and non-math mode
\newcommand{\minentropy}[1]{\ensuremath{\operatorname{H_\infty}\olrk{#1}}}
\newcommand{\fuzzyminentropy}[1]{\ensuremath{\operatorname{H^{fuzz}_{\maxdist, \infty}}\olrk{#1}}}
\providecommand{\condminentropy}[2]{\ensuremath{\operatorname{\tilde{H}_\infty}\olrk{#1|#2}}}
\newcommand{\s}{\mathcmd{s}}
\newcommand{\typo}{\mathcmd{\tilde{\m}}}
\newcommand{\typoj}[1]{\ensuremath{(\typo_{#1}, j_{#1})}}
\newcommand\typojprime[1]{{(\typo_{#1}, j'_{#1})}}
\renewcommand\contrib[2]{\mathsf{cont}\left[{#1}/{#2}\right]}
\def\typovec{\ensuremath{\{\typo_i\}_{i=1}}}
\def\opttypovec{\ensuremath{\{\typo^*_i\}_{i=1}}}
\def\slotguess{\textsf{SlotGuess}}
\def\typodist{\ensuremath{\tau_\pw}}
\def\cachedtypodist{\ensuremath{{\tilde{\tau}}_{\pw}}}
\def\typodistest{\ensuremath{{\hat{\tau}}_{\pw}}}
\def\incache{T_\pw}
\newcommand{\fuzzylambda}{\ensuremath{\lambda^{\mathrm{fuzzy}}}}
%\newcommand{\errorprob}[2]{\mathcmd{\tau_{#1}({#2})}}
\newcommand{\errordist}{\mathcmd{\tau}}
\newcommand{\famdist}{\mathcmd{\mathcal{W}}}
\def\guessw{\ensuremath{W}}
\newcommand{\precdist}{W}
\newcommand{\entdef}{Z}
\newcommand{\PW}{\mathcmd{\mathcal{W}}}
\newcommand{\cachedom}{\mathcmd{\mathcal{S}}}
\newcommand{\supp}{\mbox{supp}}
\newcommand{\olrk}[1]{\ifx\nursymbol#1\else\!\!\mskip4.5mu plus 0.5mu\left(\mskip0.5mu plus0.5mu #1\mskip1.5mu plus0.5mu \right)\fi}

\newcommand{\errorprob}[1]{\mathcmd{\tau_{#1}}}
\newcommand{\errorpr}[2]{\mathcmd{\errorprob{#1}{(#2)}}}
\newcommand{\Expectation}{\mathop{\mathbb E}}
\newcommand{\hash}[2]{\mathcmd{F(#1, #2)}}
\newcommand{\hashj}[3]{\mathcmd{F_{#3}(#1, #2)}}
\newcommand{\rhh}{\mathcmd{y}}
\newcommand{\x}{\mathcmd{x}}
\newcommand{\range}{\mathcmd{R}}
\newcommand{\lmm}{\mathcmd{l_\w}}
\newcommand{\rmm}{\mathcmd{r_\w}}
\newcommand{\conflict}[2]{\mathcmd{C_{#1, #2}}}
\newcommand{\Probsub}[2]{{\Pr_{#1}\left[\,{#2}\,\right]}}
\newcommand{\Condprobsub}[3]{{\Pr}_{#1}\left[\: #2\:\left|\right.\:#3\:\right]}
\newcommand{\wmid}{l_{\w, \w^\prime}}
\newcommand{\realhash}{\mathcmd{z}}
\newcommand{\collhash}{\mathcmd{z^\prime}}
\newcommand{\floor}[1]{\left \lfloor #1 \right \rfloor }
\newcommand{\ceiling}[1]{\left \lceil #1 \right \rceil }
\newcommand{\interval}{I}
% \newcommand{\p}[2]{p_{#1}(#2)}
\newcommand{\ssketch}{\mathcmd{\mathsf{S}}}
\newcommand{\tsketch}{\mathcmd{\msgsettingsym\mathsf{S}}}
\newcommand{\trivsketch}{\mathcmd{\tsketch_{\emptystr}}}
\newcommand{\WREC}{\mathcmd{\pcnotionstyle{W\pcmathhyphen{}REC}}}
\newcommand{\AREC}{\mathcmd{\pcnotionstyle{A\pcmathhyphen{}REC}}}
\newcommand{\WUTIL}{\mathcmd{\pcnotionstyle{W\pcmathhyphen{}UTIL}}}
\newcommand{\AUTIL}{\mathcmd{\pcnotionstyle{A\pcmathhyphen{}UTIL}}}
\newcommand{\indexset}{I}
\newcommand{\topset}{Z_1^*}
\newcommand{\cachedist}{T}
\newcommand{\pwdis}{R}
\newcommand{\onbudget}{q}
\newcommand{\blacklist}{\textsf{B}}
\newcommand{\blacklistlen}{\alpha}
\newcommand{\plaintextstate}{\bar{\state}}
\newcommand{\typocutoff}{\delta}
\newcommand{\plfuprob}{\ensuremath{\nu}}

\def\ballweight{\ensuremath{e_{\tilde{\tau}}}}
\newcommand{\Adv}{\textnormal{\textsf{Adv}}}
\newcommand{\AdvOFFLINE}[1]{\Adv^{\footnotesize\textnormal{\textrm{offdist}}}_{\footnotesize #1}}
\newcommand{\AdvOFFLINEP}[1]{\Adv^{\overline{\footnotesize\textnormal{\textrm{{offdist}}}}}_{\footnotesize #1}}
\newcommand{\AdvOFFGUESS}[1]{\Adv^{\footnotesize\textnormal{\textrm{offguess}}}_{\footnotesize #1}}
\newcommand{\AdvONGUESS}[1]{\Adv^{{\footnotesize\textnormal{\textrm{onguess}}}}_{\footnotesize #1}}
\newcommand{\AdvONATTACK}[1]{\Adv^{\footnotesize\textnormal{\textrm{onattack}}}_{\footnotesize #1}}
\newcommand{\AdvRKROR}[1]{\Adv^{\footnotesize\textnormal{\textrm{mk-ror}}}_{\footnotesize #1}}
\newcommand{\AdvSRKROR}[1]{\Adv^{\footnotesize\textnormal{\textrm{sk-ror}}}_{\footnotesize #1}}
\newcommand{\AdvROR}[1]{\Adv^{\footnotesize\textnormal{\textrm{ror}}}_{\footnotesize #1}}
\newcommand{\AdvROBUST}[1]{\Adv^{\footnotesize\textnormal{\textrm{rob}}}_{\footnotesize #1}}

\newcommand{\advantage}[3]{\pcnotionstyle{\Adv}^{#1}_{#2}(#3)}
\newcommand{\neighbourhood}[2]{N_{#1}({#2})}
\newcommand{\rhhdist}{Y}
\newcommand{\fullerhash}{z}
\newcommand{\errorball}[2]{{B}^{\errordist}_{#1}(#2)}
\newcommand{\f}{f}
\newcommand{\pwmaxlen}{\ell}
\newcommand{\indexi}{i}
\newcommand{\indexq}{q}
\newcommand{\indexx}{x}
\newcommand{\replace}{\textsf{replace}}
\newcommand{\vecca}{\vecc{a}}
\newcommand{\veccb}{\vecc{b}}
\newcommand{\veccv}{\vecc{v}}
\newcommand{\coins}{r}
\renewcommand{\state}{\ensuremath{\mathsf{s}}}
\newcommand{\win}{\mathsf{win}}
\newcommand{\Chk}{\textsf{Chk}}
\newcommand{\PChk}{\textsf{PChk}}
\newcommand{\indexh}{h}
\newcommand{\randomZ}{Z}
\newcommand{\Cache}{\textnormal{\textsf{Cache}}}
\newcommand{\WarmupCache}{\textnormal{\textsf{WarmupCache}}}
\newcommand{\CacheUpdate}{\textnormal{\textsf{CacheUpdt}}}
\newcommand{\CacheInit}{\textnormal{\textsf{CacheInit}}}
\newcommand{\cachestate}{\ensuremath{\mathtt{S}}\xspace}
\newcommand{\cachestatelen}{\ensuremath{\sigma}\xspace}
\newcommand{\CacheLRU}{\textnormal{\textsf{CacheLRU}}}
\newcommand{\CacheLRUUpdate}{\textnormal{\textsf{UpdateLRU}}}
\newcommand{\CacheLRUInit}{\textnormal{\textsf{InitLRU}}}


\newcommand{\zxcvbn}{\textnormal{\textsf{zxcvbn}}\xspace}
\newcommand{\bad}{\textnormal{\textsf{bad}}}
\newcommand{\wait}{z}
\newcommand{\PKE}{\textnormal{\textsf{PKE}}}
\newcommand{\pkegen}{\mathcal{K}}
\newcommand{\pkeenc}{\mathcal{E}}
\newcommand{\pkedec}{\mathcal{D}}
\newcommand{\pkectxtspace}{\mathcal{C}_{\pkeenc}}
\newcommand{\pk}{{pk}}
\newcommand{\sk}{{sk}}
\newcommand{\PBE}{\textnormal{\textsf{PBE}}}
\newcommand{\utility}{\textsf{Utility}}

\newcommand{\skegen}{\ensuremath{\mathsf{K}}}
\newcommand{\skeenc}{\ensuremath{\mathsf{E}}}
\newcommand{\skedec}{\ensuremath{\mathsf{D}}}
\newcommand{\skectxtspace}{\ensuremath{\mathcal{C}_{\skeenc}}}
\newcommand{\ske}{\textnormal{\textsf{SKE}}}
\newcommand{\SE}{\textnormal{\textsf{SE}}}
\newcommand{\sekg}{\mathcal{K}}
\newcommand{\seenc}{\mathcal{E}}
\newcommand{\sedec}{\mathcal{D}}
\newcommand{\saltlen}{{\ell_{\mathrm{salt}}}}

\newcommand{\skk}{\textnormal{\textsf{k}}}
\newcommand{\cipherske}{{c}}
\newcommand{\cipherpke}{{c}}
\newcommand{\SH}{\textnormal{\textsf{SH}}}
\newcommand{\FH}{\textnormal{\textsf{H}}}
\newcommand{\counter}{\textnormal{\textsf{Count}}}
\newcommand{\indexp}{p}
\newcommand{\ciphertextspace}{\mathcal{C}}
\newcommand{\states}{\mathcal{S}}
\newcommand{\indexr}{r}
\newcommand{\vecc}[1]{\mathbf{#1}}
\newcommand{\sawait}{\overline{\sa}}
\newcommand{\skkwait}{\overline{\skk}}
\newcommand{\hwait}{\overline{\indexh}}
\newcommand{\randomY}{Y}
\newcommand{\indexm}{m}
\newcommand{\hashtable}[1]{H[#1]}
\newcommand{\SHlen}{k_1}
\newcommand{\FHlen}{k_2}
\newcommand{\valid}{\textnormal{\textsf{valid}}}
\newcommand{\skepsilon}{\epsilon_{ske}}
\newcommand{\pkepsilon}{\epsilon_{pke}}
\newcommand{\hybindex}{j}

\newcommand{\DL}{\textnormal{DL}}
\newcommand{\vecb}{\mathbf{b}}
\newcommand{\indicator}{\mathbb{I}}
\newcommand{\cipherspace}{\mathcal{C}}
\newcommand{\pkemsgspace}{\mathcal{M}_\pkeenc}

\newcommand{\pkecipher}{c}
\newcommand{\skecipher}{{c}}
\newcommand{\pkemsg}{m}
\newcommand{\skemsg}{{m}}
\newcommand{\skemsgspace}{ \msgspace_{\skeenc}}
\newcommand{\upperb}{\alpha}

\newcommand{\ffbox}[1]{
   \setlength{\fboxsep}{-1\fboxrule}
   \fbox{\hspace{1.2pt}\strut#1\hspace{1.2pt}}}


\newcommand{\statespace}{S}
\newcommand{\targuess}{\textsf{TarGuess}}
\newcommand{\untarguess}{\textsf{UnTarGuess}}

\newcommand{\accept}{\texttt{accept}}
\newcommand{\reject}{\texttt{reject}}
\newcommand{\threshold}{\ensuremath{\tau}}
\newcommand{\clfthreshold}{\ensuremath{\theta}}
\newcommand{\far}{\ensuremath{\alpha}}
\newcommand{\frr}{\ensuremath{\beta}}
\newcommand{\fpr}{\ensuremath{\nu}}
\newcommand{\fnr}{\ensuremath{\gamma}}
\newcommand{\FPR}{\textrm{FPR}}
\newcommand{\FNR}{\textrm{FNR}}


\newcommand{\matchalgo}{\mathcal{L}}
\newcommand{\matchalgovec}{\mathcal{\mathbf{L}}_\sslen}
\def \score {\theta}
\newcommand{\indexjvec}{\ensuremath{\mathcmd{\mathbf{j}}\xspace}}
\def\k{k}
\newcommand{\add}{\funcfont{Add}}
\newcommand{\accuracy}{\funcfont{Accuracy}}
\def\acc{\delta}
\def \MS {\funcfont{MS}}
\def \MRec {\funcfont{MRec}}
\newcommand \MSt{\MS^{\matchalgo}_\sslen}
\newcommand \MRect{\MRec^{\matchalgo}_\sslen}
\def \nmatches {\ensuremath{l}}
\def \D{D}
\def \Dmod{\tilde{D}}
\def \mlen{n}
\def \sslen{t}
\def \secret{s}
\def \secretvec{\mathbf{s}}
\newcommand{\sketch}{{\funcfont{sketch}}}
\newcommand{\sketchval}{v}
\def \recover{\funcfont{recover}}
\def \verify{\funcfont{verify}}
\def \q{q}
\def\fp{\alpha}
\def\fn{\beta}
\def\tp{\gamma}
\newcommand{\one}{\ensuremath{\mathbbm{1}}}
\newcommand{\seclen}{\ell}
\newcommand{\Overify}{{\mathcal{O}_\funcfont{vrfy}}}
\newcommand{\Omatch}{\mathcal{O}_\funcfont{match}}
\newcommand{\ith}[2]{#1_{#2}}
\newcommand{\jth}[2]{#1^{#2}}
\newcommand{\mspacei}{\ith{\mspace}{i}}
\newcommand{\errori}{\ith{\error}{i}}
\newcommand{\matchalgoi}{\ith{\matchalgo}{i}}
\newcommand{\mvecbar}{\overline{\mvec}}
\newcommand{\mvectilde}{\tilde{\mvec}}
\newcommand{\angbrac}[1]{\ensuremath{\langle #1 \rangle}}
\newcommand{\clf}{\mathcal{C}}
\newcommand{\clft}{\clf_\sslen}
\newcommand{\clfest}{\mathcal{\hat{\clf}}}
\newcommand{\clfq}[1]{{\clf^q_{#1}}}
\newcommand{\nclf}{r}
\newcommand{\oclf}{\omega}
\newcommand{\biosketch}{\funcfont{TenSketch}\xspace}
\newcommand{\tensketch}{\funcfont{TenSketch}\xspace}

\newcommand{\hvec}{\mathbf{h}}
\newcommand{\insertr}{\mathsf{insert{\scriptstyle\$}}}
\newcommand{\dbmsg}{\ensuremath{\mathcal{F}}}
\newcommand{\dbmsgs}{\ensuremath{\{\dbmsg_i\}}}
\newcommand{\dbid}{\mathcal{I}}
\newcommand{\db}{\ensuremath{D}}
% \newcommand{\dbs}{\ensuremath(\dbid,\{\dbmsg_i\})}
\newcommand{\dbs}{\ensuremath(\dbid,\dbmsg)}
\newcommand{\dbsprime}{\ensuremath(\dbid',\dbmsg')}
\newcommand{\dbsize}{N}
\def \dblen{N}
\def\mhat{\hat{\mvec}}
\def\findmatches{\funcfont{\footnotesize FindMatches$^{\matchalgo}$}}
\newcommand{\setss}[1]{\SS^\Delta_{#1}}
\newcommand{\setrec}[1]{\Rec^\Delta_{#1}}
\newcommand{\setsst}{\setss{\sslen}}
\newcommand{\setrect}{\setrec{\sslen}}
\newcommand{\costs}{{c_s}}
\newcommand{\costc}{{c_c}}
\NewDocumentCommand{\indseq}{ O{1} O{r} }{{#1}\ldots {#2}}
\newcommand{\seq}[2]{{{#1}_1,\ldots,{#1}_{#2}}}
\newcommand{\relu}{\funcfont{ReLu}}
\newcommand{\fcrelu}[1]{\funcfont{FC}_{\relu}^{#1}}
\newcommand{\gencliq}{\funcfont{GenTupIncr}}

%%% Local Variables:
%%% mode: latex
%%% TeX-master: "main"
%%% End:
